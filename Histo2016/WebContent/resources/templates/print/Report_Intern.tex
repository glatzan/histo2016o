\documentclass{scrreprt}
\usepackage[ngerman]{babel}
\usepackage[utf8]{inputenc}
\usepackage{graphicx}
\usepackage{wallpaper}
\usepackage[absolute]{textpos}
\usepackage{tabularx} 
\usepackage[T1]{fontenc}
\usepackage{supertabular}
\usepackage[left=2cm,right=2cm,top=3cm,bottom=2cm]{geometry}
\usepackage{amssymb}
\renewcommand{\familydefault}{\sfdefault} 
\usepackage{helvet} 
\usepackage{lipsum} 
 \usepackage{longtable}


\usepackage{lastpage}
\usepackage{fancyhdr}

\pagestyle{fancy}



\setcounter{secnumdepth}{0}

\begin{document}
%\ThisCenterWallPaper{1}{D:/latex/intern_v21.pdf}
\ThisCenterWallPaper{1}{background.pdf}
%
 \input{d:/latex/code128.tex}

\cfoot{\thepage\ of \pageref{LastPage}}

\baselineskip10pt

\textblockorigin{0in}{0in}
\setlength{\TPHorizModule}{1mm}
\setlength{\TPVertModule}{1mm}


{
\footnotesize
\begin{textblock}{100}(25, 10) 
\noindent <PATname>, <PATvorname>\\
<PAanschrift>\\
<PATplz> <PATort>\\\\
\X=.35mm        % The width of code will be greater
\barheight=1cm % the height will be smaller.
\code{<PATpiz>}\\
<PATpiz>\\
\end{textblock}
}

~\\
\vspace{5.0cm}\\
{\huge Histologisches Gutachten} 

\begin{center}
{
	\renewcommand{\arraystretch}{0}
	\setlength{\tabcolsep}{.16667em}
	\begin{tabular}{p{0.53\linewidth}p{0.47\linewidth}}
	
		{
			\renewcommand{\arraystretch}{1}
			\begin{tabular}[t]{p{0.3\linewidth}p{0.67\linewidth}}
				
 				Präperat& ::material \\
 				~& asd \\
			 	~& asd \\
 				~& asd \\
 				
 				\\
 				
 				\multicolumn{2}{l}{Vorgeschichte, klinischer Befund und Fragestellung} \\
 				\huge RA &  {\includegraphics[scale=0.4]{augen.png}} \\
 				\multicolumn{2}{l}{asdasdasdsadasdasdasd} 
			
			\end{tabular}
		} 
		& 
		{
			\renewcommand{\arraystretch}{1}
			\begin{tabular}[t]{p{0.46\linewidth}p{0.5\linewidth}}
				
				Eingangsdatum & 10.10.2016 \\
				Eingangsnummer & 1234567 \\
				~ & 
					\X=.30mm        % The width of code will be greater
					\barheight=1cm % the height will be smaller.
					\code{123456} \\
				Versicherung &
					{
						\setlength{\tabcolsep}{0em}
						\begin{tabular}{c c c c }
						$\boxtimes$ &~regulär~ & $\boxtimes$ & ~privat
						\end{tabular}
					}\\
				Station & asd \\
				Maligner Tumor & $ \Box$  \\
				Operateur/Einsender & asd \\
				Augenarzt & asd
			 
			\end{tabular}
		}
		\tabularnewline
		
	\end{tabular}
}
\end{center}

\begin{center}
\setlength{\tabcolsep}{0em}
\begin{longtable}{p{\linewidth}}
	\Large  Befund \\*
	\begin{addmargin}{0.2cm}
		\normalsize \lipsum[2-4] 
	\end{addmargin}\\
	\Large Diangosen \\*
	\begin{addmargin}{0.2cm}
		\normalsize \lipsum[2-4] 
	\end{addmargin} 
\end{longtable}
\end{center}

\begin{center}
\begin{tabular}{p{0.15\linewidth}p{0.25\linewidth}p{0.3\linewidth}p{0.3\linewidth}}
	Freiburg den & 10.10.2016 & Dr HAse & Dr maus \\
	~ & ~ &  ~Arzt & ~Oberazt
\end{tabular}
\end{center}
\end{document}