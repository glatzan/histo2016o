\documentclass{scrreprt}
\usepackage[ngerman]{babel}
\usepackage[utf8]{inputenc}
\usepackage{graphicx}
\usepackage{wallpaper}
\usepackage[absolute]{textpos}
\usepackage{tabularx} 
\usepackage[T1]{fontenc}
\usepackage{supertabular}
\usepackage[left=2cm,right=2cm,top=3cm,bottom=2cm]{geometry}
\usepackage{amssymb}
\usepackage{helvet} 
\usepackage{lipsum} 
\usepackage{longtable}
\usepackage{lastpage}
\usepackage{fancyhdr}
\usepackage{ifthen}
\usepackage{setspace}
\usepackage{everypage}
\usepackage{ragged2e}

\renewcommand{\familydefault}{\sfdefault} 

\def\fs{\kern 0.33em}

% header and footer syle
\pagestyle{fancy}

% header thinkness on page on
\AddEverypageHook{
   \ifthenelse{\value{page}=1}
    {\renewcommand{\headrulewidth}{0.0pt}}
    {\renewcommand{\headrulewidth}{0.4pt}}
}

\lfoot{
	\footnotesize 
	Dies ist ein technisch validierter Befundbericht, der erst durch eine ärztliche Unterschrift fachlich validiert ist.   
}

\rfoot{
	\footnotesize 
	{Seite \thepage\ / \pageref{LastPage}}
}


\lhead{
	\footnotesize 
   	\ifthenelse{\value{page}=1}
    	{}
 	{$!patient.person.fullName - $date.format('dd.MM.yyyy', $!patient.person.birthday) - PIZ: $!patient.piz}
}


\setcounter{secnumdepth}{0}





\begin{document}
%\ThisCenterWallPaper{1}{D:/latex/intern_v21.pdf}
\ThisCenterWallPaper{1}{background_noLogo.pdf}
%
 \input{code128.tex}


\baselineskip10pt

\textblockorigin{0in}{0in}
\setlength{\TPHorizModule}{1mm}
\setlength{\TPVertModule}{1mm}


% pat block
{
\scriptsize
\begin{textblock}{100}(19, 8) 
\noindent $!patient.person.fullName\\
$date.format('dd.MM.yyyy', $!patient.person.birthday)\\
$!patient.person.contact.street \\
$!patient.person.contact.postcode $!patient.person.contact.town\\
\X=.35mm        
\barheight=1cm 
\code{#if($patient.piz == "") 0 #else $!patient.piz #end}\\
$!patient.piz\\
\end{textblock}
}

% add block
{
\begin{textblock}{100}(22, 50) 
\setstretch{1.0}
\noindent 
\textbf{$!subject} \\\\
$address \\

\end{textblock}
}

~\\
\vspace{5.0cm}\\
\noindent\makebox[\linewidth]{\rule{\linewidth}{0.4pt}}\\

{
	\RaggedRight
	\renewcommand{\arraystretch}{0}
	\setlength{\tabcolsep}{.16667em}
	\begin{tabular}{@{}p{0.6\linewidth}p{0.4\linewidth}}
	
		{
			\renewcommand{\arraystretch}{1.5}
			\begin{tabular}[t]{p{0.3\linewidth}p{0.7\linewidth}}
				
	 		
 				\multicolumn{2}{l}{\huge Histologisches Gutachten} \\ 
 				$!patient.person.fullName & geb.  $date.format('dd.MM.yyyy', $!patient.person.birthday)
			
			\end{tabular}
		} 
		& 
		{
			\renewcommand{\arraystretch}{1}
			\begin{tabular}[t]{p{0.6\linewidth}p{0.4\linewidth-8\tabcolsep}}
				Eingangsnummer & \textbf{$!task.taskID} \\
				~ & 
					\X=.30mm        % The width of code will be greater
					\barheight=1cm % the height will be smaller.
					\code{$!task.taskID} \\
			\end{tabular}
		}
	\end{tabular}

	\begin{tabular}{p{1\linewidth}}
		Sehr geehrte Frau Kollegin,\\
		sehr geehrter Herr Kollege,\\
		wir berichten über die histologische Untersuchung,\\
	\end{tabular}
	\linebreak
	\linebreak
}



{
\renewcommand{\arraystretch}{1}
\setlength{\tabcolsep}{0em}
\setlength{\LTleft}{5pt}
\setlength{\LTpre}{0pt}
\setlength{\LTpost}{0pt}
\RaggedRight

#foreach ($diagnosisRevision in $task.diagnosisContainer.diagnosisRevisions)
\large \underline{ $!diagnosisRevision.name: }\smallskip
\begin{longtable}{p{\linewidth}}
		\normalsize $!latexTextConverter.convertToTex($diagnosisRevision.text)
\end{longtable}

\ \\
\large \underline{ Diagnosen: }\smallskip
\begin{longtable}{p{0.05\linewidth}p{0.95\linewidth}}
#foreach ($diagnosis in $diagnosisRevision.diagnoses)
	$!diagnosis.sample.sampleID & $!latexTextConverter.convertToTex($diagnosis.diagnosis)  \\
#end
\end{longtable}	
			
#end
}
 
\begin{center}
\RaggedRight
\begin{tabular}{p{0.15\linewidth}p{0.05\linewidth}p{0.35\linewidth}p{0.45\linewidth}}
	\multicolumn{4}{l}{Mit freundlichen Grüßen} \\\\
	Freiburg den \linebreak $date.format('dd.MM.yyyy', $!task.diagnosisContainer.signatureDateAsDate) & & $!task.diagnosisContainer.signatureOne.physician.person.fullName #if($!task.diagnosisContainer.signatureOne.physician.person.fullName) \linebreak #end $!task.diagnosisContainer.signatureOne.role & $!task.diagnosisContainer.signatureTwo.physician.person.fullName #if($!task.diagnosisContainer.signatureTwo.physician.person.fullName) \linebreak #end $!task.diagnosisContainer.signatureTwo.role \\


\end{tabular}

\end{center}
   \ifthenelse{\pageref{LastPage}>1}
     { \noindent \footnotesize  Dieser Befundbericht umfasst insgesamt \pageref{LastPage} Seiten.  \\}
      {~	}
\end{document}